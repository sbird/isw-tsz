\documentclass[aps,twocolumn,floats,prd,nofootinbib]{revtex4-1}


\usepackage[dvips]{graphicx} % 
\usepackage{graphicx,amsmath,amsfonts,amssymb,slashed}
%\usepackage{graphicx,amsmath,amsfonts,amssymb,aas_macros,slashed}
\usepackage{bbold,wasysym}
\usepackage{graphicx}

\usepackage[usenames,dvipsnames]{xcolor} 

\usepackage{soul}


\definecolor{RedWine}{rgb}{0.743,0,0}
\definecolor{RoyalBlue}{rgb}{0.25,.41,.88}


\setstcolor{Blue}


\def\jpcomment#1{\textcolor{Blue}{\bf [JP: #1]}}
\def\jp#1{\textcolor{Blue}{  #1}}
\def\mk#1{\textcolor{Red}{  #1}}
\def\todo#1{\textcolor{blue}{[{\sc #1}]}}




\newcommand{\years}{\ensuremath{\mathrm{y}}}
\renewcommand{\sec}{\ensuremath{\mathrm{s}}}

\newcommand{\kg}{\ensuremath{\mathrm{kg}}}

\newcommand{\mbarn}{\ensuremath{\mathrm{mb}}}
\newcommand{\barn}{\ensuremath{\mathrm{b}}}

\newcommand{\fm}{\ensuremath{\mathrm{fm}}}
\newcommand{\cm}{\ensuremath{\mathrm{cm}}}
\newcommand{\km}{\ensuremath{\mathrm{km}}}
\newcommand{\Mpc}{\ensuremath{\mathrm{Mpc}}}

\newcommand{\eV}{\ensuremath{\mathrm{eV}}}
\newcommand{\keV}{\ensuremath{\mathrm{keV}}}
\newcommand{\MeV}{\ensuremath{\mathrm{MeV}}}
\newcommand{\GeV}{\ensuremath{\mathrm{GeV}}}
\newcommand{\TeV}{\ensuremath{\mathrm{TeV}}}


\newcommand{\keVee}{\ensuremath {\mathrm{keVee}}}
\newcommand{\keVnr}{\ensuremath {\mathrm{keVnr}}}
\newcommand{\Er}{\ensuremath {E_R}}
\newcommand{\Ev}{\ensuremath {E_\mathrm{v}}}
\newcommand{\cpd}{\ensuremath {\mathrm{cpd}}}
\newcommand{\days}{\ensuremath {\mathrm{days}}}
\newcommand{\ton}{\ensuremath {\mathrm{ton}}}
\renewcommand{\day}{\ensuremath {\mathrm{day}}}

\newcommand{\DM}{\ensuremath {\mathrm{DM}}}
\newcommand{\SM}{\ensuremath {\mathrm{SM}}}

\newcommand{\Aprime}{\ensuremath V}
\newcommand{\mv}{\ensuremath m_V}
\newcommand{\SMgroup}{\ensuremath {\mathrm SU}(3)_{c}\times {\mathrm SU}(2)_{L}\times{\mathrm U}(1)_{Y}}
\newcommand{\Uoneprime}{\ensuremath {\mathrm U}(1)_{V}}
%\newcommand{\kappa}{\ensuremath \kappa}

\newcommand{\meff}{\ensuremath {m_{\mathrm{eff}}}}
\def\VEV#1{\left\langle #1 \right\rangle}


%\newcommand{\Pr}{\ensuremath {\rm Pr}}


\DeclareMathOperator{\real}{Re}
\DeclareMathOperator{\imag}{Im}
\DeclareMathOperator{\Tr}{Tr}
\DeclareMathOperator{\erf}{Erf}
\DeclareMathOperator{\erfi}{Erfi}
\DeclareMathOperator{\Br}{Br}
\DeclareMathOperator{\Floor}{Floor}


\newcommand{\dth}[3]{\ensuremath{ \left( \frac{\partial #1 }{\partial #2}\right)_{#3}}}
\newcommand{\dthflat}[3]{\ensuremath{ ( \partial #1 / \partial #2)_{#3}}}
\newcommand{\jac}[4]{\ensuremath{ \left( \frac{ \partial (#1 , #2 ) }{\partial (#3,#4) }\right)}}
\newcommand{\jacflat}[4]{\ensuremath{  \partial (#1,#2) / \partial (#3,#4) }}




\begin{document}

\title{Cross-correlation between thermal Sunyaev-Zeldovich
     effect and the integrated Sachs-Wolfe effect}
\author{Cyril Creque-Sarbinowski$^1$, Simeon Bird$^2$, and Marc
     Kamionkowski$^2$} 
\affiliation{$1$Department of Physics, Massachusetts Institute
     of Technology, 77 Massachusetts Avenue, Cambridge, MA 02139, USA\\
     $^2$Department of Physics and Astronomy, Johns Hopkins
     University, 3400 N.\ Charles Street, Baltimore, MD 21218, USA}

\begin{abstract}
Large-angle fluctuations in the cosmic microwave background
(CMB) temperature induced by the integrated Sachs-Wolfe (ISW)
effect and Compton-$y$ distortions from the thermal
Sunyaev-Zeldovich (tSZ) effect are both due to line-of-sight
density perturbations.  Here we calculate the cross-correlation
between these two signals.  Measurement of this
cross-correlation can be used to test the redshift distribution
of the tSZ distortion.  We also evaluate the detectability of a
$yT$ cross-correlation from exotic early-Universe sources in the
presence of this late-time effect.
\end{abstract}
%\date{\today}


\maketitle
\section{Introduction}
\label{sec:intro}

The standard $\Lambda$CDM cosmological model provides a
remarkably good fit to an array of precise measurements.
However, there still remain some tensions between different
measurements which must be resolved, and the physics responsible
for the generation of primordial perturbations has yet to be
delineated.  This paper addresses both these issues.

Large-angle fluctuations in the cosmic microwave background
(CMB) temperature ($T$) are induced not only by density perturbations
at the CMB surface of last scatter (the Sachs-Wolfe effect; SW), but
also by the growth of density perturbations along the line of
sight (the integrated Sachs-Wolfe effect; ISW)
\cite{Sachs:1967er}.  Although the
CMB frequency spectrum is very close to a blackbody, there are
small distortions, of the Compton-$y$ type, induced by the rare
scattering of CMB photons from hot
electrons in the intergalactic medium (IGM) of galaxy clusters
\cite{Sunyaev:1972eq}.  This $y$ distortion has been mapped, as
a function of position on the sky, by Planck with an angular
resolution of a fraction of a degree
\cite{Ade:2013qta,Aghanim:2015eva}, and there are vigorous
discussions of future missions, such as PIXIE \cite{Kogut:2011xw}
and PRISM \cite{Andre:2013nfa}, that will map the $y$ distortion
with far greater sensitivity and resolution.

Given that both the tSZ and ISW fluctuations are induced by
density perturbations at relatively low redshifts, there should
be some cross-correlation between the two \cite{Taburet:2010hb},
and the purpose of
this paper is to calculate this cross-correlation.  The
motivation for this work is two-fold:  First, there is some
tension between the measured amplitude of $y$ fluctuations and
the amplitude of density perturbations inferred from CMB
measurements
\cite{Lueker:2009rx,Komatsu:2010fb,Ade:2013lmv,Ade:2015fva}.
The tension, though, is based
upon theoretical models that connnect the $y$-distortion and
density-perturbation amplitudes.  Ingredients of these models
include nonlinear evolution of primordial perturbations, gas
dynamics, and feedback processes, all of which can become quite
complicated.  Any empirical handle on this physics would
therefore be useful.

The second motivation involves the search for exotic
early-Universe physics.  Recent work has shown that primordial
non-gaussianity may lead to a $yT$ cross-correlation
which may be used to probe scale-dependent non-gaussianity
\cite{Emami:2015xqa}.  The present calculation will be used to
explore whether this early-Universe signal can be distinguished
from late-time effects that induce a $yT$ correlation.


\section{Calculation}
\label{sec:calculation}

\subsection{The ISW Effect}

The integrated Sachs-Wolfe (ISW) effect describes the frequency
shift of CMB photons as they traverse through time-evolving
gravitational potentials.  The fractional temperature
fluctuation in a direction $\hat n$ due to this frequency shift
is 
\begin{equation}
     \frac{\Delta T}{T}(\hat{n}) =
     -\frac{2}{c^2}\int d\eta\,
     \frac{d\phi}{d\eta}(c\eta \hat n;z)
     = -\frac{2}{c^2}\int\, dz 
     \frac{d\phi}{dz}(r \hat n;z) 
\label{eqn:isw}
\end{equation}
where $\phi(\vec x,z)$ is the gravitational potential at
position $\vec x$ and redshift $z$, $\eta$ the conformal time,
$c$ the speed of light, and $r$ the distance along the line of
sight.

The potential $\phi$ is related to the density perturbation
through the Poisson equation $\nabla^2 \phi = 4 \pi G \rho$,
where $\nabla$ is a gradient with respect to physical position,
$G$ Newton's constant, and $\rho$ the matter density. We write
$\rho(\vec x;z)= \bar\rho [1+\delta(\vec x;z)]$ in terms of the
mean matter density $\bar \rho$ and fractional density
perturbation $\delta(\vec x;z)$.  We then use the Friedmann
equation to write $\bar \rho = (3/8\pi G) \Omega_m H_0^2 a^{-3}$ in
terms of the matter density $\Omega_m$ (in units of the critical
density), Hubble parameter $H_0^2$, and scale factor
$a=(1+z)^{-1}$.  We further write the density perturbation
$\delta(\vec x;z) = D(z) \delta(\vec x;z=0)$ in terms of the
linear-theory growth factor $D(z)$  We can then re-write the
Poisson equation as,
\begin{equation}
     \phi(\vec x;z) = -\frac{3}{2} \Omega_m H_0^2
     \frac{D(z)}{a(z)} \left[ \nabla_c^{-2} \delta(\vec x;z=0)
     \right],
\label{eqn:phik}
\end{equation}
where $\nabla_c=\nabla/a$ is the gradient with respect to
comoving coordinate.

The power spectrum for ISW-induced angular temperature
fluctuations is then obtained using the Limber approximation,
which can be stated as follows:  If we observe a two-dimensional
projection,
\begin{equation}
     p(\hat n) = \int \, dr\, q(r) \delta(r \hat n),
\end{equation}
of a three-dimensional field $\delta(\vec x)$, with
line-of-sight-distance weight function $q(r)$, then the angular
power spectrum, for multipole $l$, of $p(\hat n)$ is
\begin{equation}
     C_l^p = \int \, dr\, \frac{[q(r)]^2}{r^2} P(l/r),
\label{eqn:Limber}
\end{equation}
in terms of the three-dimensional power spectrum $P(k)$, in
terms of wavenumber $k$, for $\delta(\vec x)$.

Using Eqs.~(\ref{eqn:isw}), (\ref{eqn:phik}), and
(\ref{eqn:Limber}), we find the power spectrum for ISW-induced
temperature fluctuations to be,
\begin{eqnarray}
     C_l & = & \left(\frac{3 \Omega_m H_0^2}{c^3} \right)^2
     \frac{c}{l^4} \int r^2  dr \left[ H(z) \frac{d}{dz}
     \left( \frac{D(z)}{a(z)} \right) \right]^2 P\left(\frac{l}{r}
     \right) \nonumber \\ 
        & = & \left(\frac{3 \Omega_m H_0^2}{c^3} \right)^2
     \frac{c}{l^4} \int  dz H(z) \left[r(z) \frac{d}{dz}
     \left( \frac{D(z)}{a(z)} \right) \right]^2 P\left(\frac{l}{r}
     \right), \nonumber \\
     & = &  \int \frac{c\, dz}{H(z)} \left[ \Delta^{\rm
     isw}_l(z)\right]^2 P(l/r),
\label{eqn:Clisw}
\end{eqnarray}
in terms of the matter-density power spectrum $P(k)$ today.
Note that we used the relation $dz = -(H/c) dr$ to get from the
first to the second line in Eq.~(\ref{eqn:Clisw}, and we have
defined in the last line the ISW transfer function,
\begin{equation}
     \Delta^{\rm isw}_l(z) = \frac{3 \Omega_m H_0^2}{c^3 l^2}
     r(z) H(z) \frac{d}{dz}
     \left( \frac{D(z)}{a(z)} \right).
\label{eqn:iswtransfer}
\end{equation}


\subsection{The Thermal Sunyaev-Zeldovich Effect}

The thermal SZ effect (tSZ) arises from inverse-Compton
scattering from the hot electrons in the intergalactic medium of
galaxy clusters. This upscattering induces a frequency-dependent
shift in the CMB intensity in direction $\hat n$ which we write
as a brightness-temperature fluctuation,
\begin{align}
     \left(\frac{\Delta T}{T}\right)_\nu(\hat{n}) = g(\nu) y \equiv
     \left(x\frac{e^x + 1}{e^x - 1} - 4\right)y(\hat{n}),
\end{align}
where $y(\hat n)$ is the $y$ distortion in direction $\hat n$,
and $x \equiv h\nu/k_B T$, with $\nu$ the frequency, $k_B$ the
Boltzmann constant, $h$ the Planck constant, and $T=2.727$~K the
CMB temperature.  The Compton-$y$ distortion is given by an integral,
\begin{align}
     y(\hat{n}) &\equiv \frac{k_B\sigma_T}{m_e
     c^2}\int\,  ds \, n_e(s \hat n) T_e
     (s \hat n),
\label{eqn:yparameter}
\end{align}
along the line of sight, where $s$ is the (physical) line-of-sight
distance, $\sigma_T$ the Thomson cross section, $n_e(\vec x)$ the
electron number density at position $\vec x$, and $T_e(\vec x)$
the electron temperature.  The hot electrons that give rise to
this distortion are assumed to be housed in galaxy clusters
with a variety of masses $M$ and a variety of redshifts $z$.
The spatial abundance of clusters with masses between $M$ and
$M+dM$ at redshift $z$ is $(dn/dM)dM$ in terms of a mass
function $(dn/dM)(M,z)$, a function of mass and redshift.
Galaxy clusters of mass $M$ at redshift $z$ are distributed
spatially with a fractional number-density perturbation that is
assumed to be $b(M,z)\delta(\vec x)$ in terms of a bias
$b(M,z)$.  The spatial fluctuations to the electron pressure
$P_e(\vec x)=k_B n_e(\vec x) T_e(\vec x)$ that give rise to
angular fluctuations in the Compton=$y$ parameter induced by
clusters of mass $M$ and redshift $z$ can then be
modeled as $b(M,z)$ times a convolution of the density
perturbation $\delta(\vec x)$ with the electron-pressure profile
of the cluster.  Since convolution in configuration space
corresponds to multiplication in Fourier space, the Limber
derivation discussed above can be used to find the power
spectrum for angular fluctuations in the Compton-$y$ parameter
to be \cite{Komatsu:1999ev,Diego:2004uw,Taburet:2010hb,Ade:2013qta,Ade:2015mva}
\begin{equation}
     C_l^{yy,2h} = \int \frac{c\, dz}{H(z)} \left[
     \Delta^{y}_l(z) \right]^2
     P(l/r),
\label{eqn:Cly2h}
\end{equation}
in terms of a transfer function,
\begin{equation}
     \Delta_l^{y}(z) = r(z) D(z) \int \frac{dn}{dM}dM 
     y_l(M,z) b(M,z).
\end{equation}
Here, $y_l(M,z)$ is the 2d Fourier transform of the
Compton-$y$ image, on the sky, of a cluster of mass $M$ at
redshift $z$ and is given in terms of the electron pressure
profile $P_e(M,z;x)$, as a function of scale radius $x$ in the
cluster, for example, in Eq.~(A.5) in Ref.~\cite{Aghanim:2015eva}.
We use for our numerical work the electron-pressure profiles of
Ref.~\cite{Komatsu:2002wc}.

The ``2h'' superscript in the
$y$-parameter power spectrum indicates that this is the
``two-halo'' contribution, the $y$ autocorrelation that arises
from large-scale density perturbations.  There is an additional
``one-halo'' contribution that arises from Poisson fluctuations
in the number of clusters.  This is
\cite{Komatsu:2002wc},
\begin{equation}
     C_l^{yy,1h} = \int\, dz \left[r(z)\right]^2\frac{c}{H(z)}
     \int dM \frac{dn(M,z)}{dM} \left| y_l(M,z)
     \right|^2.
\label{eqn:Cly1h}
\end{equation}
The total $y$-parameter power spectrum is $C_l^{yy} =
C_l^{yy,1h}+C_l^{yy,2h}$.  

\subsection{ISW-tSZ cross-correlation}

Given that the temperature fluctuation induced by the ISW effect
and the two-halo contribution to tSZ fluctuations are both
generated on large angular scales by the same fractional density
perturbation $\delta(\vec x)$, there should be a
cross-correlation between the two.  From the expressions,
Eq.~(\ref{eqn:Clisw}) and (\ref{eqn:Cly2h}, it is clear that this
cross-correlation is
\begin{equation}
     C_l^{yT} = \int \frac{c\, dz}{H(z)} \Delta^{\rm isw}_l(z)
     \Delta^y_l(z) P(l/r).
\label{eqn:Clcross}
\end{equation}

\begin{figure}[h]
%\includegraphics[width=10 cm, height = 7 cm]{CL.png}
\caption{The $y,2h$ and ISW power spectrum, along with the
     $y,2h$-$ISW$ cross correlation are drawn in solid lines,
     while the $y, 1h$ is dotted. The corresponding early
     universe power spectra are shown in dashed lines. We place
     the CMB power spectrum for comparison.} 
\label{fig:power}
\end{figure}

\subsection{Numerical results and approximations}

Fig.~\ref{fig:power} shows the resulting power spectra.  
For our numerical results, we use a vacuum-energy density (in
units of critical) $\Omega_{\Lambda} = 0.721$, matter density
$\Omega_m = 0.279$, baryon density $\Omega_b = 0.046$, a
critical density for collapse of $\delta_c = 1.686$, and
dimensionless Hubble parameter $h= 0.701$, although the
large-angle results that will be our primary focus are largely
insensitive to these details.  In practice, we find that the
majority of the contribution comes from a redshift integration
from $z=0$ to $z=4$ and a halo-mass integration between
$10^{12}\,M_\odot$ and $10^{15}\,M_\odot$.

The large angle (low-$l$) behaviors of the ISW-ISW
autocorrelation, the $yT$ cross-correlation, and the one- and
two-halo contributions to the $yy$ power spectra are easy to
understand qualitatively.  Let us begin with the ISW effect.  Here, the
$l$ dependence of the transfer function is $\Delta^{\rm isw}_l
\propto l$, and for large angles ($l\lesssim 100$), the power
spectrum is $P(l/r) \propto l$.  As a result, $l^2C_l^{\rm isw}
\propto l^{-1}$ for $l \lesssim 100$.  Next consider the tSZ
power spectra.  Galaxy cluster subtend a broad distribution of
angular sizes but are rarely wider than a degree.  Thus, for $l
\lesssim 100$, they are effectively point sources.  The
Fourier transform is thus effectively approximated by $y_l(M,z)
\simeq y_{l=0}(M,z)$ which is itself precisely the integral of
the $y$-distortion over the cluster image on the sky, or
equivalently, the total contribution of the cluster to the
angle-averaged $y$.  As a result of the independence of $y_l$ on
$l$ and $P(l/r) \propto l$ for $l\gtrsim100$, we infer
$l^2 C_l^{yy,2h} \propto l$ and $l^2 C_l^{yT} \propto $const for
$l\lesssim 100$.  Finally, the one-halo contribution to
$C_l^{yy}$ is nearly constant (i.e., $l^2C_l\propto l^2$) for
$l\lesssim 100$ as expected for Poisson fluctuations in what are
(at these angular scales) effectively point sources.

\section{SZ redshift distribution}

We now discuss the prospects to learn about the redshift
distribution of the galaxy clusters that produce the Compton-$y$
distortion.  As seen above, the $yT$ correlation is significant
primarily at multipole moments $l \lesssim 100$, where the
window function $y_l(z)$ is largely independent of $l$.  The
amplitude of the cross-correlation, relative to the
auto-correlations, can be understood largely by examining the
overlap between the redshift dependences of the two transfer
functions $\Delta_l^y(z)$ and $\Delta_l^{\rm isw}(z)$.  These
transfer functions are shown in Fig.~\ref{fig:Deltas}.  More
precisely, we plot--noting that $P(l/r) \propto l/r$ for the
relevant angular scales---$\Delta_l/\left[H(z) r(z) \right]^{1/2}$,
the square root of the integrands for $C_l$, as it is the
overlap of these two functions that determines the strength of
the cross=correlation relative to the auto-correlation.  We also
normalize the curves in Fig.~\ref{fig:Deltas} to both have the
same area under the curve.

\begin{figure}[h]
%\includegraphics[width=10 cm, height = 7 cm]{CL.png}
\caption{We plot the transfer functions $\Delta_l^{\rm isw}(z)$
     and $\Delta_l^y(z)$, divided by $[r(z)H(z)]^{1/2}$. The
     squares of the plotted quantities are the redshift ($z$)
     integrands for the ISW power spectrum $C_l^{\rm isw}$ and
     the two-halo contribution to the tSZ power spectrum
     $C_l^{yy}$.  Both curves are normalized so that the
     areas under the curve are the same.  The tSZ-ISW
     cross-correlation $C_l^{yT}$ is obtained from the overlap
     of these two.}
\label{fig:Deltas}
\end{figure}






Given the current fairly precise constraints to dark-energy
parameters, the predictions for $\Delta_l^{\rm isw}(z)$ has
relatively small uncertainties.  The prediction for
$\Delta_l^y(z)$ depends, however, on the redshift distribution
of the halo mass function, bias parameters, and cluster pressure
profiles, all of which involve quite uncertain physics.
Measurement of the $yT$ correlation will, however, provide an
additional empirical constraint to the redshift evolution of the
$y$ parameter.

To see how this might work, we replace
\begin{equation}
     \Delta_l^y(z) \to \Delta_l^y(z) \left[ 1 +
     \epsilon (z-z_0) \right],
\label{eqn:replacement}
\end{equation}
where
\begin{equation}
     z_0 = \frac{ \int \frac{dz}{r(z)H(z)} z \Delta_l^y(z)}
     { \int \frac{dz}{r(z)H(z)} \Delta_l^y(z)} \simeq ????.
\label{eqn:z0}     
\end{equation}
The functional form in Eq.~(\ref{eqn:replacement}), is chosen so
that, with $z_0$ given in Eq.~(\ref{eqn:z0}), the
auto-correlation power spectrum $C_l^{yy}$ will remain unaltered
for small $\epsilon$.  This alteration thus describes, for
$\epsilon>0$, a weighting of the Compton-$y$ distribution to
smaller redshifts (and {\it vice versa} for $\epsilon<0$) in
such a way that leaves the total $y$ signal unchanged.

We now estimate the smallest value $\sigma_\epsilon$ of
$\epsilon$ that will be detectable with future measurements.
This is given by
\begin{equation}
     \frac{1}{\sigma_\epsilon}^2 \simeq \sum_l \frac{ \partial
     C_l^{yT}/\partial \epsilon}{\left(\sigma_l^{yT} \right)^2},
\end{equation}
where
\begin{equation}
     \frac{\partial C_l^{yT}}{\partial \epsilon} = \int
     \frac{c\, dz}{H(z)} \Delta^{\rm isw}_l(z) \Delta^y_l(z)
     (z-z_0) P(l/r).
\label{eqn:Clderiv}
\end{equation}

\section{Primordial Non-Gaussianity}

We now review the $yT$ cross-correlation from the
scale-dependent primordial non-gaussianity scenario of
Ref.~\cite{Emami:2015xqa}.  If primordial perturbations are
non-gaussian, the amplitude of small-wavelength power can be
modulated by long-wavelength Fourier modes of the density
field.  The dissipation of primordial Fourier modes with
wavenumbers $k\simeq 1-50$ Mpc$^{-1}$ (which takes place at
redshifts $1100 \lesssim z \lesssim 5\times 10^4$) give rise to
primordial Compton-$y$ distortions.  If there is non-gaussianiy,
then the angular distribution of this $y$ distortion may be
correlated with the large-scale density modes that give rise,
through the Sachs-Wolfe effect, to large-angle fluctuations in
the CMB temperature.

The predictions for this primordial $yT$ correlation depend on the
yet-unmeasured isotropic value $\VEV{y}$ of the Compton-$y$
parameter for which we take as a canonical value $4\times
10^{-9}$.  The $yy$ and $yT$ power spectra for the scenario are
then, 
\begin{eqnarray}
     l^2 C_l^{yy,{\rm ng}} &\simeq& 5.5\times10^{-20} \, \left( \frac{
     f_{\rm nl}^y}{200} \right)^2 \left(\frac{\VEV{y}}{4\times
     10^{-9}} \right)^2, \\
     l^2 C_l^{yT,{\rm ng}} &\simeq& 5.8\times10^{-15} \, \left( \frac{
     f_{\rm nl}^y}{200} \right) \left(\frac{\VEV{y}}{4\times
     10^{-9}} \right).
\end{eqnarray}
Here, $f_{\rm nl}^y$ is the non-gaussianity parameter for
squeezed bispectrum configurations in which the wavenumber of
the long-wavelength mode is of the $\sim$Gpc$^{-1}$ scales of
modes that contribute to the ISW effect, while the two
short-wavelength modes have wavelengths $1\,{\rm Mpc}^{-1}
\lesssim k \lesssim 50\,{\rm Mpc}^{-1}$
As discussed in Ref.~\cite{Emami:2015xqa}, there are no
existing model-independent constraints to $f_{\rm nl}^y$.

We now estimate the detectability of the $yT$ cross-correlation
from non-gaussianity, discussed in Ref.~\cite{Emami:2015xqa}.
In that work, the late-time contribution to $C_l^{yy}$ and
$C_l^{yT}$ was neglected, and the detectability of the
primordial signal inferred assuming that detection of $y$
fluctuations was noise-limited.  Here we re-do those estimates
taking into account the late-time $yT$ correlation calculated
above.  Under the null hypothesis of no early-Universe
contribution to the $yT$ correlation, the error with which
each $C_l^{yT}$ can be determined is
\begin{equation}
     \left(\sigma_l^{yT} \right)^2 =\frac{1}{2l+1} \left[ \left(
     C_l^{yT} \right)^2 + C_l^{TT}\left(
     C_l^{yy} + N_lW_l^{-2} \right)\right], 
\end{equation}
where $C_l^{TT}$ is the CMB temperature power spectrum,
$W_l = e^{-l^2\sigma_b^2/2}$ is a window function, and $ N_l
= (4\pi/N)\sigma_y^2$ is the noise in the measurement of
$C_l^{yy}$.  Here, $\sigma_b$ the beam size and, $\sigma_y$ the
root-variance of the $y$-distortion measurement in each
pixel, and $N$ the number of pixels.

The Planck satellite has now measured the tSZ power spectrum and
found good agreement with the expectations from the one-halo
contribution to $C_l^{yy}$.  They have now even presented good
evidence for detection of the two-halo contribution at
$l\lesssim 10$.  From this we infer that the noise contribution
$N_l$ to the $yy$ measurement is already small compared with
$C_l^{yy}$, and it will be negligible for future experiments
like PIXIE or PRISM.  We also note from the numerical results
that $\left(C_l^{yT}\right)^2$ is small compared with $C_l^{yy}
C_l^{yT}$---this makes sense given that the cross-correlation of
$y$ with the ISW effect is small and further that the ISW effect
provides only a small contribution to large-angle temperature
fluctuations.  We may thus approximate
\begin{equation}
     \left(\sigma_l^{yT} \right)^2 \simeq \frac{1}{2l+1}
     C_l^{TT} C_l^{yy}.
\label{eqn:noiseapprox}     
\end{equation}

The signal-to-noise with which an early-Universe
$yT$ signal with power spectrum $C_l^{yT,{\rm ng}}$ can be
distinguished from the null hypothesis is
\begin{equation}
     \left(\frac{S}{N} \right) = \left(\sum_l\frac{\left(C_l^{yT,{\rm ng}}
     \right)^2}{(\sigma_l^{yT})^2}\right)^{1/2},
\label{eqn:SN}
\end{equation}
Using Eq.~(\ref{eqn:noiseapprox}) and the numerical results for
$C_l^{yy}$, we infer a signal-to-noise $(S/N) \simeq  (f_{\rm
nl}^y/1.5\times 10^4) (\VEV{y}/4\times 10^{-9})$.  We thus see
that the late-time contribution to Compton-$y$ fluctuations
degrades the detectability $f_{\rm nl}^y$ by almost two orders
of magnitude relative to what was estimated in
Ref.~\cite{Emami:2015xqa}.  The detectabilty is, moreover,
limited by cosmic variance and not from measurment noise.
The summand in Fig.~\ref{fig:power} is dominated by $l\lesssim
10$, and so the conclusion does not depend on the assumed
angular resolution.

\section{Conclusion}

Here we have calculated the tSZ-ISW cross-correlation,
investigated its use in constraining the redshift distribution
of $y$-parameter fluctuations, and evaluated the detectability
of an early-Universe $yT$ cross-correlation.  We showed that
measurement of the $yT$ cross-correlation can be used to
constrain the redshift distribution of the sources of
$y$-parameter fluctuations, something that may be of utility
given uncertainties in the cluster-physics and
large-scale-structure ingredients (pressure profiles, halo
biases, mass functions) that determine these fluctuations.  We
also showed that estimates, that neglect the $yT$
correlations induced at late times, of the detectability of
early-Universe $yT$ correlations may be optimistic by several
orders of magnitude.

\subsection*{Acknowledgments}
 
We thank Liang Dai, Yacine Ali-Ha\"{i}moud, and Ely
Kovetz for useful discussions.  SB was supported by NASA through
Einstein Postdoctoral Fellowship Award Number PF5-160133.  This
work was supported by NSF Grant No. 0244990, NASA NNX15AB18G,
the John Templeton Foundation, and the Simons Foundation.














\begin{acknowledgments}
This work was supported at JHU by NSF Grant No.\
0244990, NASA NNX15AB18G, the John Templeton Foundation, and the
Simons Foundation. 
\end{acknowledgments}


\begin{thebibliography}{99}

%\cite{Sachs:1967er}
\bibitem{Sachs:1967er} 
  R.~K.~Sachs and A.~M.~Wolfe,
  %``Perturbations of a cosmological model and angular variations of the microwave background,''
  Astrophys.\ J.\  {\bf 147}, 73 (1967)
  [Gen.\ Rel.\ Grav.\  {\bf 39}, 1929 (2007)].
  %%doi:10.1007/s10714-007-0448-9
  %%CITATION = %doi:10.1007/s10714-007-0448-9;%%
  %1229 citations counted in INSPIRE as of 15 Apr 2016


%\cite{Sunyaev:1972eq}
\bibitem{Sunyaev:1972eq} 
  R.~A.~Sunyaev and Y.~B.~Zeldovich,
  %``The Observations of relic radiation as a test of the nature of X-Ray radiation from the clusters of galaxies,''
  Comments Astrophys.\ Space Phys.\  {\bf 4}, 173 (1972).
  %139 citations counted in INSPIRE as of 15 Apr 2016


%\cite{Ade:2013qta}
\bibitem{Ade:2013qta} 
  P.~A.~R.~Ade {\it et al.} [Planck Collaboration],
  %``Planck 2013 results. XXI. Power spectrum and high-order statistics of the Planck all-sky Compton parameter map,''
  Astron.\ Astrophys.\  {\bf 571}, A21 (2014)
  % %doi:10.1051/0004-6361/201321522
  [arXiv:1303.5081 [astro-ph.CO]].
  %%CITATION = %doi:10.1051/0004-6361/201321522;%%
  %71 citations counted in INSPIRE as of 15 Apr 2016


%\cite{Aghanim:2015eva}
\bibitem{Aghanim:2015eva} 
  N.~Aghanim {\it et al.} [Planck Collaboration],
  %``Planck 2015 results. XXII. A map of the thermal Sunyaev-Zeldovich effect,''
  arXiv:1502.01596 [astro-ph.CO].
  %%CITATION = ARXIV:1502.01596;%%
  %31 citations counted in INSPIRE as of 15 Apr 2016


%\cite{Kogut:2011xw}
\bibitem{Kogut:2011xw} 
  A.~Kogut {\it et al.},
  %``The Primordial Inflation Explorer (PIXIE): A Nulling Polarimeter for Cosmic Microwave Background Observations,''
  JCAP {\bf 1107}, 025 (2011)
  % %doi:10.1088/1475-7516/2011/07/025
  [arXiv:1105.2044 [astro-ph.CO]].
  %%CITATION = %doi:10.1088/1475-7516/2011/07/025;%%
  %182 citations counted in INSPIRE as of 15 Apr 2016


%\cite{Andre:2013nfa}
\bibitem{Andre:2013nfa} 
  P.~Andr\'e {\it et al.} [PRISM Collaboration],
  %``PRISM (Polarized Radiation Imaging and Spectroscopy Mission): An Extended White Paper,''
  JCAP {\bf 1402}, 006 (2014)
  %%doi:10.1088/1475-7516/2014/02/006
  [arXiv:1310.1554 [astro-ph.CO]].
  %%CITATION = %doi:10.1088/1475-7516/2014/02/006;%%
  %72 citations counted in INSPIRE as of 15 Apr 2016


%\cite{Taburet:2010hb}
\bibitem{Taburet:2010hb} 
  N.~Taburet, C.~Hernandez-Monteagudo, N.~Aghanim, M.~Douspis and R.~A.~Sunyaev,
  %``The ISW-tSZ cross correlation: ISW extraction out of pure CMB data,''
  Mon.\ Not.\ Roy.\ Astron.\ Soc.\  {\bf 418}, 2207 (2011)
  %doi:10.1111/j.1365-2966.2011.19474.x
  [arXiv:1012.5036 [astro-ph.CO]].
  %%CITATION = %doi:10.1111/j.1365-2966.2011.19474.x;%%
  %9 citations counted in INSPIRE as of 15 Apr 2016


%\cite{Lueker:2009rx}
\bibitem{Lueker:2009rx} 
  M.~Lueker {\it et al.},
  %``Measurements of Secondary Cosmic Microwave Background Anisotropies with the South Pole Telescope,''
  Astrophys.\ J.\  {\bf 719}, 1045 (2010)
  %doi:10.1088/0004-637X/719/2/1045
  [arXiv:0912.4317 [astro-ph.CO]].
  %%CITATION = %doi:10.1088/0004-637X/719/2/1045;%%
  %128 citations counted in INSPIRE as of 15 Apr 2016


%\cite{Komatsu:2010fb}
\bibitem{Komatsu:2010fb} 
  E.~Komatsu {\it et al.} [WMAP Collaboration],
  %``Seven-Year Wilkinson Microwave Anisotropy Probe (WMAP) Observations: Cosmological Interpretation,''
  Astrophys.\ J.\ Suppl.\  {\bf 192}, 18 (2011)
  %doi:10.1088/0067-0049/192/2/18
  [arXiv:1001.4538 [astro-ph.CO]].
  %%CITATION = %doi:10.1088/0067-0049/192/2/18;%%
  %5496 citations counted in INSPIRE as of 15 Apr 2016


%\cite{Ade:2013lmv}
\bibitem{Ade:2013lmv} 
  P.~A.~R.~Ade {\it et al.} [Planck Collaboration],
  %``Planck 2013 results. XX. Cosmology from Sunyaev–Zeldovich cluster counts,''
  Astron.\ Astrophys.\  {\bf 571}, A20 (2014)
  %doi:10.1051/0004-6361/201321521
  [arXiv:1303.5080 [astro-ph.CO]].
  %%CITATION = %doi:10.1051/0004-6361/201321521;%%
  %252 citations counted in INSPIRE as of 15 Apr 2016


%\cite{Ade:2015fva}
\bibitem{Ade:2015fva} 
  P.~A.~R.~Ade {\it et al.} [Planck Collaboration],
  %``Planck 2015 results. XXIV. Cosmology from Sunyaev-Zeldovich cluster counts,''
  arXiv:1502.01597 [astro-ph.CO].
  %%CITATION = ARXIV:1502.01597;%%
  %75 citations counted in INSPIRE as of 15 Apr 2016


%\cite{Emami:2015xqa}
\bibitem{Emami:2015xqa} 
  R.~Emami, E.~Dimastrogiovanni, J.~Chluba and M.~Kamionkowski,
  %``Probing the scale dependence of non-Gaussianity with spectral distortions of the cosmic microwave background,''
  Phys.\ Rev.\ D {\bf 91}, no. 12, 123531 (2015)
  %doi:10.1103/PhysRevD.91.123531
  [arXiv:1504.00675 [astro-ph.CO]].
  %%CITATION = %doi:10.1103/PhysRevD.91.123531;%%
  %9 citations counted in INSPIRE as of 15 Apr 2016


%\cite{Komatsu:1999ev}
\bibitem{Komatsu:1999ev} 
  E.~Komatsu and T.~Kitayama,
  %``Sunyaev - zel'dovich fluctuations from spatial correlations between clusters of galaxies,''
  Astrophys.\ J.\  {\bf 526}, L1 (1999)
  %doi:10.1086/312364
  [astro-ph/9908087].
  %%CITATION = %doi:10.1086/312364;%%
  %116 citations counted in INSPIRE as of 15 Apr 2016


%\cite{Diego:2004uw}
\bibitem{Diego:2004uw} 
  J.~M.~Diego and S.~Majumdar,
  %``The Hybrid SZ power spectrum: Combining cluster counts and SZ fluctuations to probe gas physics,''
  Mon.\ Not.\ Roy.\ Astron.\ Soc.\  {\bf 352}, 993 (2004)
  %doi:10.1111/j.1365-2966.2004.07989.x
  [astro-ph/0402449].
  %%CITATION = %doi:10.1111/j.1365-2966.2004.07989.x;%%
  %12 citations counted in INSPIRE as of 15 Apr 2016


%\cite{Ade:2015mva}
\bibitem{Ade:2015mva} 
  P.~A.~R.~Ade {\it et al.} [Planck Collaboration],
  %``Planck 2013 results. XXXII. The updated Planck catalogue of Sunyaev-Zeldovich sources,''
  Astron.\ Astrophys.\  {\bf 581}, A14 (2015)
  %doi:10.1051/0004-6361/201525787
  [arXiv:1502.00543 [astro-ph.CO]].
  %%CITATION = %doi:10.1051/0004-6361/201525787;%%
  %7 citations counted in INSPIRE as of 15 Apr 2016


%\cite{Komatsu:2002wc}
\bibitem{Komatsu:2002wc} 
  E.~Komatsu and U.~Seljak,
  %``The Sunyaev-Zel'dovich angular power spectrum as a probe of cosmological parameters,''
  Mon.\ Not.\ Roy.\ Astron.\ Soc.\  {\bf 336}, 1256 (2002)
  %doi:10.1046/j.1365-8711.2002.05889.x
  [astro-ph/0205468].
  %%CITATION = %doi:10.1046/j.1365-8711.2002.05889.x;%%
  %265 citations counted in INSPIRE as of 15 Apr 2016


\end{thebibliography}

\end{document}



%\cite{Sheth:1999mn}
\bibitem{Sheth:1999mn} 
  R.~K.~Sheth and G.~Tormen,
  %``Large scale bias and the peak background split,''
  Mon.\ Not.\ Roy.\ Astron.\ Soc.\  {\bf 308}, 119 (1999)
  %doi:10.1046/j.1365-8711.1999.02692.x
  [astro-ph/9901122].
  %%CITATION = %doi:10.1046/j.1365-8711.1999.02692.x;%%
  %1524 citations counted in INSPIRE as of 15 Apr 2016


%\cite{Bryan:1997dn}
\bibitem{Bryan:1997dn} 
  G.~L.~Bryan and M.~L.~Norman,
  %``Statistical properties of x-ray clusters: Analytic and numerical comparisons,''
  Astrophys.\ J.\  {\bf 495}, 80 (1998)
  %doi:10.1086/305262
  [astro-ph/9710107].
  %%CITATION = %doi:10.1086/305262;%%
  %1033 citations counted in INSPIRE as of 15 Apr 2016


%\cite{Mo:1995cs}
\bibitem{Mo:1995cs} 
  H.~J.~Mo and S.~D.~M.~White,
  %``An Analytic model for the spatial clustering of dark matter halos,''
  Mon.\ Not.\ Roy.\ Astron.\ Soc.\  {\bf 282}, 347 (1996)
  %doi:10.1093/mnras/282.2.347
  [astro-ph/9512127].
  %%CITATION = %doi:10.1093/mnras/282.2.347;%%
  %927 citations counted in INSPIRE as of 15 Apr 2016



     


where $\delta({\vec k},z)$ is the first-order fractional density
perturbation for wavevector $\vec k$, $z$ is the redshift,
$\Omega_m$ the matter density in units of the critical density,
$H_0$ the present-day Hubble parameter, and $a(z)$ the scale
factor (normalized to unity today).  The time (i.e., redshift)
dependence of the density perturbation is $\vec k$ independent
and is $\delta({\vec k},z) = D(z)\delta(\vec k,z=0)$.
By plugging the inverse Fourier transform of
Eq.~(\ref{eqn:phik}) into Eq.~(\ref{eqn:isw}) and changing the
integration variable from conformal time to redshift, we obtain,
\begin{eqnarray}
     \frac{\Delta
     T}{T}(\hat{n})&=&-\frac{3H_0^2}{c^3}\Omega_m\int_{0}^{r_{\text{max}}}\,
     dr\, H(z)\frac{d}{dz}\left(\frac{D(z)}{a(z)}\right)
     \nonumber \\
     & & \times \int \frac{d^3k}{(2\pi)^3}
     e^{-i \vec{k}\cdot\vec{r}}\frac{\delta(\vec k,0)}{k^2},
\label{eqn:iswcomplete}
\end{eqnarray}
where $r_{\text{max}} = c a(0) (\eta_0 - \eta_r)= c(\eta_0 -
\eta_r)$.  The redshift $z$ in Eq.~(\ref{eqn:iswcomplete}) is
related to the distance $r$ by integrating $(dz/dr)= - H(z)/c$,
where $H(z)= H_0 E(z) = H_0 \left[\Omega_m (1+z)^3 +(1-\Omega_m)
\right]^{1/2}$ is the Hubble parameter at redshift $z$ (and here
we assumed the dark energy to be a cosmological constant).

Using the plane-wave expansion  and the addition theorem
for spherical harmonics, we find the spherical-harmonic
coefficients for Eq.~(\ref{eqn:iswcomplete}) to be,
\begin{align}
     a_{lm}^{\text{isw}} &=
     4\pi(-i)^l\int
     \frac{d^3k}{(2\pi)^3}Y^*_{lm}(\hat{k})
     \Delta_l^{\text{ISW}}(k)\delta(\vec k,0),
\label{eqn:almisw}
\end{align}
where
\begin{align}
     \Delta_l^{\text{isw}}(k) &\equiv
     -\frac{3H_0^2}{c^3}\Omega_m\int_{0}^{r_{\text{max}}}\,dr\,
     H(z)\frac{d}{dz}\left(\frac{D(z)}{a(z)}\right)\frac{j_l(k
     r)}{k^2}.
\end{align}










before. In order to calculate the pressure $P_e(\vec x)=n_e(r_p)k_B
T_e(r_p)$, we write Eq.~(\ref{eqn:yparameter}) as a sum over the
pressure profiles of galaxy clusters that cross the line of
sight; i.e., \mk{Check this equation!}
\begin{equation}
     P_e(\vec x)  = \sum_{i} \frac{m_e c^2}{
     \sigma_T}y_{3D}\left( | \vec x -\vec w_i| \right),
\end{equation}
where $y_{3D}(w)$ is the weighted radial
pressure profile from Eq.~(7) in Ref.~\cite{Komatsu:2002wc}, and
$\vec w_i$ is the position of the center of cluster $i$.
We assume all clusters have the same profile and so replace $w$ with
$\vec{r}_p-\vec{w}_n$, where $\vec{w}_n$ is the position of the
center of the $n^{\text{th}}$ dark-matter halo.  We also swap
the sum over $\Delta n$, which is equal to unity, for an
integral over $dn$ and introduce the galaxy-cluster mass
function [Eq.~(10) in Ref.~\cite{Sheth:1999mn}] $\frac{dn}{dM
dV}(M,z,\vec{w}_p)$. The $z$ appearing in the argument of our
functions are all related to $r_p$. Through this process we have
also gone from the discrete line-of-sight variable $\vec{w}_n$
to the continuous $\vec{w}_p$.   Now due to the presence of
$\vec{r}_p - \vec{w}_p$ in the pressure profile, we replace the
spatial cluster mass function and pressure profile with their
expressions in the physical Fourier domain, as convolution in
the spatial domain is multiplication in the Fourier domain. This
leads to the following expression for the temperature
flunctuations:
\begin{align}
     \frac{\Delta T}{g(\nu)T_{\text{CMB}}}&(\hat{n}) = \int
     \frac{dr_p dM(a(z))^{-3}d^3k_c}{(2\pi)^3}
     e^{-i\vec{k_c}\cdot\vec{r}_c} \nonumber \\
     &
     \times\frac{dn}{dMdV}(M,z)b(M,z)D(z)\delta_{k_p}(0)
     \nonumber \\
     &  \times \widetilde{y}_{3D}\left(\frac{k_c}{a(z)},M,z\right).
     \nonumber 
\end{align}
The expression for the Fourier transform of the radial profile
is
\begin{align}
     \widetilde{y}_{3D}(\vec{k}_p,M,z) &= 4\pi
     \int_{0}^{r_v(M,z)}dr_p\ y(r_p,M,z)j_0(\vec{k}_p\cdot
     \vec{r}_p)r_p^2.
\end{align}
We cut off at the virial radius, 
\begin{equation}
     r_v(M,z) = \sqrt[3]{\frac{3M}{4\pi \Delta_c(z) \rho_c(z)}},
\end{equation}
with the spherical overdensity, $\Delta_{c} (z)$, from Eq.~(6)
in Ref.~\cite{Bryan:1997dn} and the critical overdensity
$\rho_c(z)$ at redshift $z$.

The Fourier transform of the cluster mass function is
\begin{eqnarray}
     \frac{dn}{dM}(M,z,\vec{k}_p) &\simeq&
     \frac{dn}{dMdV}(M,z)\delta_{k_p}(z)b(M,z) \nonumber \\
     & & + (2\pi)^3 \frac{dn_{d}}{dMdV}(M,z) \delta_{D}^{(3)}
     (\vec{k}_p),
\end{eqnarray}
where $b(M,z)$ is the bias from Ref.~\cite{Mo:1995cs}.  The
first term in this expression accounts for the 2-halo
contribution to the x-ray-gas correlation---i.e., the
correlation that arises between two different galaxies.  The
second term is the one-halo contribution which accounts for
x-ray-gas correlations that arise from gas in the same halo.

For the 2-halo case term,, we change to comoving coordinates,
noting that  $\delta_{k_p}(z) = a^3(z)\delta_{k_c}(z)$, and then
find from the projection of the plane wave onto spherical
harmonics the spherical-harmonic coefficients,
\begin{equation}
     a_{lm}^{y,2h} = 4\pi(-i)^l\int
     \frac{d^3k_c}{(2\pi)^3}Y^{*}_{lm}(\hat{k}_c)
     \Delta_l^{y,2h}(k_c)\delta_{k_c}(0),
\end{equation}
where
\begin{eqnarray}
     \Delta_l^{y,2h}(k_c) &\equiv& \int dr dM
     \frac{a(z)dn}{dMdV}(M,z)b(M,z) D(z) \nonumber \\
  & &   \times
  \widetilde{y}_{3D}\left(\vec{k}_p,M,z\right)j_l(k_c r).
\end{eqnarray}
The 1-halo contribution is derived differently since there
is no plane wave with which to expand.  Instead, we can use the
Limber approximation in the Fourier transform of the gas
profile, that $k_p \approx \frac{l+\frac{1}{2}}{a(z)r (z)}$, in
order to pull it out of the integral over $d^3k_p$ so as to obtain
\begin{eqnarray}
     a_{lm}^{y,1h} &= &\int d^3 r_p dM \frac{Y^{*}_{lm}(\hat{n})}
     {r_p^2} \frac{dn_{d}}{dMdV}(M,z) \nonumber \\
   & & \times
   \widetilde{y}_{3D}\left(\frac{l+\frac{1}{2}}{a(z)r_c(z)},M,z\right) 
\end{eqnarray}
